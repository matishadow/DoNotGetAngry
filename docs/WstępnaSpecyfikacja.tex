\documentclass[12pt]{article}
\usepackage{hyperref}
\usepackage[utf8]{inputenc}
\usepackage[T1]{fontenc}

\title{{\Huge \vspace{-4cm} \textbf{"Do not get angry."}} \\ {\Large Wstępna specyfikacja projektu}}
\author{Dominika Kucharska \\ Mateusz Tracz \\ Jakub Serwatka}
\date{}
\pagenumbering{gobble}

\begin{document}
\maketitle
\vspace{1mm}

\section{Opis projektu}
Komputerowa wersja popularnej gry planszowej "Chińczyk" (ang. "Do not get angry"). \\
Gracze po połączeniu się z serwerem gry tworzą nowy pokój lub dołączają do już istniejącego. \\
Gdy w pokoju znajdują się przynajmniej dwie osoby, właściciel pokoju ma możliwość rozpoczęcia rozgrywki. \\
Zasady gry odpowiadają zasadom obowiązującym w wersji planszowej. 

\section{Typ gry}
Gra przeznaczona dla 2-4 graczy.

\section{Mechanizm komunikacji}
Gniazda TCP

\section{Język programowania}
Python 3.6.0+

\section{Biblioteki pomocnicze}
Pygame (silnik graficzny)

\section{Link do repozytorium}
\href{https://github.com/matishadow/DoNotGetAngry}{Link Github}. 


\end{document}